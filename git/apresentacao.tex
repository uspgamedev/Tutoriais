\documentclass[brazil]{beamer}
\usepackage{beamerthemesplit}
\usepackage[brazilian]{babel}
\usepackage[utf8]{inputenc}
\usepackage{hyperref}
\usepackage{color}
\usepackage{xcolor}
\usepackage{amssymb}
\usepackage{amsmath}
\usepackage{fancybox}
\usepackage{ulem}
\usepackage{listings}
\usetheme{JuanLesPins}

\title{Curso de Introdução ao Git}
\author{USPGameDev}

\begin{document}

\frame{\titlepage}

%-------------------------------------
\section{1. Introdução}
%-------------------------------------

\frame{\begin{center}
\Huge Introdução
\end{center}}

\frame{
  \frametitle{O que é Controle de Versão?}
  Softwares de controle de versão é um sistema que grava todas as modificações em um conjunto de arquivos, criando um 
  histórico deles, criando um ambiente onde o usuário possa recriar versões antigas dos arquivos.
  \\
  Eles são separados em 3 conjuntos:
  \begin{itemize}
    \item Controle de versão local. (Ex: rcs)
    \item Controle de versão centralizado. (Ex: SVN, CVS)
    \item Controle de versão distribuído. (Ex: Git, Mercurial)
  \end{itemize}
}

%-------------------------------------
\subsection{Fundamentos}
%-------------------------------------

\frame{\begin{center}
\Huge Fundamentos
\end{center}}

\frame{
  \frametitle{Snapshots, não diferenças}
  O Git, diferente de outros controles de versões que salvam os arquivos e as modificações entre os commits, 
  o Git salva, a cada commit, o estado atual de cada arquivo.
}

\frame{
  \frametitle{Trabalho Local}
  Quase todo trabalho do Git é local. Isso significa que ele não precisa de comunicação constante com algum servidor
  remoto, como é normal em outros CVs. Essa liberdade garante que os usuários não fiquem dependentes de internet ou 
  intranet para trabalhar.
}

\frame{
  \frametitle{Integridade}
  Todo arquivo que está sob os cuidados do Git recebe um checksum ( soma de verificação ). Olhando para esse checksum,
  o Git sabe se algum arquivo ou diretório foi modificado, tornando impossível modificações obscuras.
}

\frame{
  \frametitle{Adição de dados (geralmente...)}
  Quase todas as ações do Git tem como finalidade adicionar dados na database do projeto. Depois que algo foi 
  incorporado em um repositório, é quase impossível tirá-lo de lá.
}

\frame{
  \frametitle{3 Estados}
  Todo e qualquer arquivos que está no Git tem 3 possíveis estados:
  \begin{itemize}
    \item Committed - Significa que as modanças feitas no arquivo estão salvas na database do Git.
    \item Modificado - Significa que o arquivo foi modificado.
    \item Staged - Significa que as mudanças atuais no arquivos seram inclusas no próximo commit.
  \end{itemize}
}

\frame{
  \frametitle{3 Estados}
  %Enfiar imagem aqui.
  Esses 3 estados fazem com que um repositório local do Git seja separado em 3 seções:
  \begin{itemize}
    \item Git directory - Onde todos os arquivos vitais e a database do Git estão.
    \item Working directory - É uma instância de uma versão do seu projeto.
    \item Staging area - Um arquivo que contêm todas as modificações que entrarão no próximo commit.
  \end{itemize}
}

\frame{
  \frametitle{3 Estados}
  O processo de trabalho com o Git pode ser resumido a:
  \begin{enumerate}
    \item Você modifica os arquivos no seu working directory.
    \item Você adiciona as modificações na staging area.
    \item Você commita, aicionando permanentemente as modificações no seu Git directory. 
  \end{enumerate}
}

%-------------------------------------
\subsection{Instalação e Configuração}
%-------------------------------------

\frame{
  \frametitle{Instalação}
  Para instalar o Git, vamos seguir o tutorial do GitHub =D
}

\begin{frame}[fragile]
  \frametitle{Configuração}
  O Git permite uma variedade de configurações. Podemos modificá-las usando o comando:
  \begin{block}{}
    \verb#$ git config#
  \end{block}
  \pause
  Duas configurações que temos que modificar depois de instalar o Git são:
  \begin{block}{}
      \verb#$ git config --global user.name "José da Silva"# \\
      \verb#$ git config --global user.email josé@silva.com.br#
  \end{block}
  Temos que mudar essas configurações pois o Git usam elas para gerar as mensagens de commit.
\end{frame}

%-------------------------------------
\section{Git 101}
%-------------------------------------

%-------------------------------------
\subsection{Repositórios}
%-------------------------------------

\begin{frame}[fragile]
  \frametitle{Iniciando um Repositório}
  Podemos pegar um projeto que usa o Git de duas maneiras distintas:
  \begin{itemize}
    \item Portanto um projeto existente para o Git, ou
    \item Clonando um projeto que já usa o Git.
  \end{itemize}
\end{frame}

\begin{frame}[fragile]
  \frametitle{Iniciando um Repositório}
  Para iniciar o Git num projeto já existente basta ir na pasta raiz do projeto e usar o comando:
  \begin{block}{}
    \verb#$ git init#
  \end{block}
  \pause
  Para clonar um projeto que já usa o Git usamos o comando:
  \begin{block}{}
    \verb#$ git clone <url-do-repositorio> <nome-da-pasta>#
  \end{block}
  Esse comando cria uma pasta com o nome do projeto no local em que foi chamado. Podemos
  passar um nome para a pasta que vai seer criada.
\end{frame}

%-------------------------------------
\subsection{Gravando Mudanças}
%-------------------------------------

\begin{frame}[fragile]
  \frametitle{Tracked e Untracked}
  Agora que temos um repositório funcionando, vamos entender como o Git grava as mudanças. \\
  \pause
  Lembram que todo arquivo no Git tinha 3 estados? \\
  \pause
  Podemos adicionar mais dois estados gerais, \textbf{Tracked} e \textbf{Untracked}. \\
  \pause
  Os arquivos \textbf{Tracked} estão na database do Git, e eles podem estar em qualquer dos 3 estados citados
  anteriormente. \\
  \pause
  Os \textbf{Untracked} são os arquivos que não estão na database do Git, ou seja, o Git não grava as mudanças
  feitas neles.
\end{frame}

\begin{frame}[fragile]
  \frametitle{Tracked e Untracked}
  %Imagem tracked untracked%
\end{frame}

\begin{frame}[fragile]
  \frametitle{Git status}
  Mas como sabemos em qual estado estão os arquivos? \\
  \pause
  Podemos usar o comando status do Git, que tem essa saída:
  \begin{block}{}
    \begin{verbatim}
      $ git status
      # On branch master
      # Changes to be committed:
      #   (use "git reset HEAD <file>..." to unstage)
      #
      #   new file:   README
      #
    \end{verbatim}
  \end{block}
  Veremos o que a saída significa depois.
\end{frame}

\begin{frame}[fragile]
  \frametitle{Git add}
  O comando git add tem duas principais funcionalidades:
  \begin{itemize}
    \item Adicionar um novo arquivo na database do Git. O arquivo passa de Untracked para Tracked.
    \item Adicionar uma nova versão de um Arquivo Tracked na database. O arquivo passa de modified para staged.
  \end{itemize}
  A sintaxe é simples:
  \begin{block}{}
    \verb#$ git add <nome-do-arquivo>#
  \end{block}
\end{frame}

\begin{frame}[fragile]
  \frametitle{Mais git status}
  Agora vamos analisar a saída do comando status...
  \begin{block}{}
    \begin{verbatim}
      $ git status
      # On branch master
      # Changes to be committed:
      #   (use "git reset HEAD <file>..." 
              to unstage)
      #
      #   new file:   README
      #
      # Changed but not updated:
      #   (use "git add <file>..." 
              to update what will be committed)
      #
      #   modified:   benchmarks.rb
    \end{verbatim}
  \end{block}
\end{frame}

\begin{frame}[fragile]
  \frametitle{.gitignore}
  Nem sempre 
\end{frame}

\end{document}
