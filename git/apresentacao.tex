\documentclass[brazil]{beamer}
\usepackage{beamerthemesplit}
\usepackage[brazilian]{babel}
\usepackage[utf8]{inputenc}
\usepackage{color}
\usepackage{xcolor}
\usepackage{amssymb}
\usepackage{amsmath}
\usepackage{fancybox}
\usepackage{ulem}
\usepackage{listings}
\usetheme{JuanLesPins}

\title{Curso de Introdução ao Git}
\author{USPGameDev}

\begin{document}

\frame{\titlepage}

%-------------------------------------
\section{1. Introdução}
%-------------------------------------

\frame{
  \frametitle{O que é Controle de Versão?}
  Softwares de controle de versão é um sistema que grava todas as modificações em um conjunto de arquivos, criando um 
  histórico deles, criando um ambiente onde o usuário possa recriar versões antigas dos arquivos.
  \\
  Eles são separados em 3 conjuntos:
  \begin{itemize}
    \item Controle de versão local. (Ex: rcs)
    \item Controle de versão centralizado. (Ex: SVN, CVS)
    \item Controle de versão distribuído. (Ex: Git, Mercurial)
  \end{itemize}
}

\end{document}
