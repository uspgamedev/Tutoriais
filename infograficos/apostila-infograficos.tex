
\documentclass[12pt,onecolumn]{article}
\usepackage[brazilian]{babel}
\usepackage[utf8]{inputenc}
\usepackage{hyperref}
\usepackage[section]{placeins}
\usepackage{graphicx}
\usepackage{caption}
\usepackage{subcaption}
\usepackage{float}
\usepackage{framed,color}
\usepackage{wrapfig}

\begin{document}

% Title page.
\begin{titlepage}

    % Title info.
    \title{
        \bf
        \LARGE Apostila de \\
        \Huge  Infográficos
    }
    
    \author{Renan Teruo Carneiro \\ Wilson Kazuo Mizutani}
    
    % Print title.
    \maketitle
    
    % No numbering on this page.
    \thispagestyle{empty}
    
\end{titlepage}

\begin{center}
  Copyright (C) 2013 USPGameDev
\end{center}
\begin{figure}[ht]
  \centering
  \includegraphics[width=\textwidth]{CC-BY.png}
\end{figure}

\vspace{300pt}

Escrito por:
\begin{itemize}
  \item \textbf{Renan Teruo Carneiro} \textit{(imano\_ob at uspgamedev.org)}
  \item \textbf{Wilson Kazuo Mizutani} \textit{(kazuo at uspgamedev.org)}
\end{itemize}

% Page break.
\clearpage

% Print table of contents.
\tableofcontents

% Page break.
\clearpage

\section{Introdução}
  Infográficos são representações gráficas e visuais de informação, dados ou
  conhecimento. Seu objetivo é apresentar informações complexas
  eficientemente\footnotemark.
  
  \footnotetext{
    \url{http://en.wikipedia.org/wiki/Infographic}, acesso em 21/04/2013.
  }
  
  Na realidade, Infográficos já existem há muitos anos. Mas foi só recentemente
  que a proliferação de ferramentas simples e livres trouxeram a criação de
  Infográficos para o alcance de um segmento maior da socedade, princiaplmente
  no âmbito digital. É no uso de algumas dessas ferramentas que essa apostila se
  concentra. Não é nosso objetivo dar curso sobre coleta e processamento de
  dados, mas sim sobre as possíveis maneiras de se apresentá-los em seu blog ou
  site de maneira simples e elegante.
  
  Mais particularmente, trabalharemos usando o software livre WordPress como
  ferramenta de blog para publicar os infográficos gerados pelas ferramentas que
  apresentaremos.

\clearpage
\section{\textit{Embedding}}
  Antes de mostrar as ferramentas de infográficos, vamos discutir como fazer
  o restultado aparecer em um site nosso. Para isso, usa-se uma técnica chamada
  \textit{Embedding}, que se traduz mais ou menos como "incorporar". E, de fato,
  o que essa técnica faz é incorporar em um site mídia externas a ele.
  
  \subsection{Por que precisamos disso?}
    Com o conhecimento técnico apropriado, é perfeitamente possível fazer o
    infográfico que eu quiser em meu site. No entanto, isso não é nem um pouco
    prático. Seriam semanas - se não meses - de trabalho como desenvolver alguns
    poucos tipos de infográficos. Então, ao invés de reinventar a roda, vamos
    usar o ferramental disponível na Web.

    Usaremos infográficos gerados por sites especializados. Uma vez criados, os
    infográficos ficam hospedados nesses sites, disponíveis para usarmos quando
    e onde quisermos. Porém, como serão normalmente infográficos relativamente
    interativos, fazer um simples \textit{download} de uma imagem deles não
    seria exatamente o que queremos. O que queremos é que eles sejam acessados
    como parte do \emph{nosso} próprio site, e não no site que o gerou.
    
    É isso que se chama \emph{incorporar} elementos externos no nosso site. E é
    por isso que usamos essa técnica.
  \subsection{Como isso funciona?}
    Talvez você já saiba, mas todas páginas Web possuem um código fonte em
    \emph{HyperText Markup Language} (HTML). Tudo que aparece em uma página
    tem um código HTML por trás. Logo, se queremos colocar um infográfico em
    uma página, precisamos de algum tipo de código HTML.
    
    É bem mais simples do que parece, porque não precisamos fazer esse código
    nós mesmos. São as ferramentas de infografia Web responsáveis pelo nosso
    infográfico que fornecem o código necessário para nós. Isso é possível pois,
    como falamas, são esses sites especializados que hospedam o infográfico (da
    mesma maneira que o YouTube hospeda nossos e muitos outros vídeos). Basta
    copiar o trecho de código HTML fornecido pelo site e colá-lo no código do 
    \emph{nosso} site, e pronto!
    
  \subsection{Onde eu acesso o código HTML do meu site? Como eu faço um site?}
    Normalmente, contrata-se um \textit{WebMaster} para criar uma página Web, e
    ele fornece um meio de inserirmos código HTML nele - seja pedindo para ele o
    fazer por nós, seja ele nos concedendo acesso remoto ao servidor. Mas, para
    simplificar essa apostila, e manter o foco no que realmente interessa,
    usaremos a ferramenta de blog WordPress, que faz tudo isso para a gente e
    permite escrevermos o código HTML que quisermos a partir do nosso próprio
    brwoser.
    
    Nesse caso, você tem duas opções: criar um blog hospedado em
    \url{www.wordpress.com}, ou pedir para os monitores do curso criarem um
    para você\footnotemark. Uma vez que você tenha seu próprio WordPress, você
    poderá inserir código HTML nos seus \textit{posts} do blog, apresentando os
    infográficos que quiser. Os detalhes específicos de como cada ferramenta de
    infográfico é incorporado ao site são descritos nas seções a seguir.
    
    \footnotetext{
      Normalmente no endereço \url{www.uspgamedev.org/cursos/wordpress/}.
    }

\clearpage
\section{Ferramentas de uso geral}
  \subsection{Datawrapper}
     \textbf{Endereço: \url{http://datawrapper.de/}}
    
    %%Intro
    O Datawrapper é um serviço web voltado para a criação e manutenção de
    infográficos dos mais diversos tipos, relativamente interativos e com
    aparência profissional. Seu uso não exige nenhum tipo de conhecimento de
    design, apenas uma noção elementar de manipulação de planilhas. É possível
    se cadastrar no site e usar os serviços como disponibilizados, ou instalar
    sua própria instância do Datawrapper para maior personalização dos
    infográficos. Não cobriremos essas opções extras aqui, porém.
    %%Uso
    
    \begin{figure}[H]
      \begin{center}
        \includegraphics[scale=0.5]{dados-calc.png}
        \caption{Dados na planilha}
        \label{fig:dados-calc}
      \end{center}
    \end{figure}
    
    Tendo uma conta no site, para criar um infográfico, basta clicar no botão 
    "Create a Chart". O site segue um processo passo-a-passo para você elaborar
    o infográfico que desejar. O primeiro passo é entrar com os dados. Os dados
    podem ser um arquivo .csv (\emph{Comma Separated Values}) ou dados copiados
    e colados a partir de um editor de planilhas, como o \emph{Microsoft Excel}
    ou o \emph{LibreOffice Calc}. Veja um exemplo na Figura \ref{fig:dados-calc}.
    
    %%Exemplo de dados
    %%Tela com o resultado
    
    \begin{figure}[H]
      \begin{center}
        \includegraphics[scale = 0.5]{datawrapper-data.png}
        \caption{Dados reconhecidos pelo Datawrapper}
        \label{fig:datawrapper-data}
      \end{center}
    \end{figure}
    
    No segundo passo, você poderá rever os dados para confirmar se eles foram
    reconhecidos corretamente, como na Figura \ref{fig:datawrapper-data}. Você
    também pode informar ao serviço se a primeira linha e colunas são apenas
    etiquetas (ou \emph{labels}) para os dados, se as séries de dados
    encontram-se nas linhas ou nas colunas, e algumas outras opções de
    formatação nesse ponto. Também é possível creditar a fonte original dos
    dados, ou mesmo visualizá-los imediatamente.
    
    
    \begin{figure}[H]
      \begin{center}
        \includegraphics[scale = 0.5]{datawrapper-describe.png}
        \caption{Menu de descrição dos dados}
        \label{fig:datawrapper-describe}
      \end{center}
    \end{figure}
    
    Feito isso, você deve estar na parte de visualização. Primeiro você deve escolher
    o tipo de gráfico a ser usado (Colunas, linhas, pie charts, etc). Se os dados não
    estiverem na disposição esperada, aqui é possível transpor os dados (ou seja, se
    você errou na hora de informar a disposição das séries de dados, dá pra arrumar 
    aqui). Aqui também é onde se nomeia o gráfico, e onde se coloca a descrição do
    mesmo. É possível também destacar uma série de dados em particular. Por fim, 
    é nessa parte em que você pode escolher as cores usadas, além de diversas outras
    opções visuais do gráfico. Essas opções estão todas disponibilizadas no menu
    lateral, como mostrado na Figura \ref{fig:datawrapper-describe}.
    
    \begin{figure}[H]
      \begin{center}
        \includegraphics[scale=0.4]{datawrapper-charts.png}
        \caption{Opções de gráfico do Datawrapper}
        \label{fig:datawrapper-charts}
      \end{center}
    \end{figure}
    
    Finalmente, temos uma visualização do gráfico pronto no próximo passo. Agora
    falta apenas publicar. Isso é feito na última etapa, onde você tem o gráfico
    final em exibição, um link para você voltar a acessar o gráfico e o código
    HTML para \emph{embedding} do mesmo.
    
    \begin{figure}[H]
      \begin{center}
        \includegraphics[width=.9\linewidth]{datawrapper-final.png}
        \caption{Resultado final}
        \label{fig:datawrapper-final}
      \end{center}
    \end{figure}
    
    Se quiser, você também pode modificar os gráficos já publicados. Eles se encontram
    em "My Charts". Lembre-se de republicá-los caso tenha modificado alguma coisa.
    %%Exemplo?

  \begin{figure}[H]
    \begin{center}
      \includegraphics[scale=0.5]{datawrapper-republish}
      \caption{Botão de republicar o gráfico}
      \label{fig:datawrapper-republish}
    \end{center}
  \end{figure}

  \subsection{Infogram}
    \textbf{Endereço: \url{http://infogr.am/}}
    
    Infogram é um site de serviço Web voltado para a criação e manutenção de
    infográficos dos mais diversos tipos, relativamente interativos e com
    aparência profissional. Seu uso não exige nenhum tipo de conhecimento de
    design, apenas uma noção elementar de manipulação de planilhas. O site
    fornece um cadastro gratuito, com possibilidade de adquirir uma conta
    ``Pro'' para ter acesso a funcionalidades avançadas.
    
    Uma vez feita uma conta, é possível começar imediatamente a fazer
    infográficos. Na verdade, o site permite você criar um \textit{layout}
    bastante completo contendo um ou mais infográficos, textos, citações, imagens,
    mapas e vídeos. Também existem vários temas para esses \textit{layouts}.
    
    Na primeira vez que você entrar, o Infogr.am mostrará um esquema rápido
    explicando o que é cada ferramenta da interface dele. Não são muitas, é bem
    simples de entender. Elas estão divididas em quatro partes:
    
    \begin{itemize}
      \item
        \textbf{Menu lateral esquerdo:}
        Controle da conta, acesso à biblioteca pessoal de
        infográficos.
      \item
        \textbf{Menu lateral direito:}
        Adiciona elementos no \textit{layout} que você quer
        publicar, em particuar cria os infográficos.
      \item
        \textbf{Menu superior:}
        Configurações do \textit{layout}, prévia visual do resultado,
        \textit{download} (funcionalidade para contas Pro) e publicação. É nessa
        última opção (escrita como ``\textit{share}'') que você poderá obter o
        código HTML para incorporar sua produção no seu site.
      \item
        \textbf{Editor visual do \textit{layout}:}
        Aqui você pode posicionar os elementos do seu \textit{layout}, escrever
        textos neles e colocar uma nota de \textit{Copyright}. Quando você clicar
        duas vezes em um infográfico, um menu de edição aparecerá para você mudar
        as configurações e os dados usados por ele.
    \end{itemize}
    
    Quando você cria um novo infográfico no seu \textit{layout}, você pode
    escolher dentre os diversos tipos que o Infogr.am fornece, como gráficos de
    linha, gráficos de setor, gráficos de barra, gráficos de espalhamento e
    muito outros. Alguns possuem variações, e todos podem ter suas cores
    modificadas. Essa última opção e outras aparecem no menu de edição do
    infográfico, que, como dito acima, aparece quando você clica duas vezes
    nele.
    
    No menu de edição do infográfico, você pode colocar os dados que o Infogr.am
    precisa para gerá-lo e controlar outras opções do visual em geral da
    produção. Essas opções dependem do tipo de infográfico. Fora isso, você
    também pode criar mais conjuntos de dados - e o infográfico gerado terá uma
    opção interativa de alternar entre os conjuntos de dados apresentados -,
    carregar dados a partir de arquivos em seu computador, e alterar o tamanho e
    as cores do infográfico apresentado.
    
    \begin{figure}[ht]
      \centering
      \includegraphics[width=.9\linewidth]{infogram-sample-spreadsheet.png}
      \caption{
        \footnotesize
        \it
        Exemplo de planilha que o Infogr.am usa para gerar gráficos de linha.
      }
      \label{fig:infogram-sample-spreadsheet}
    \end{figure}
    
    Os dados usados pelos infográficos devem ser fornecidos em uma planilha, o que
    é bastante conveniente. No entanto, esses dados precisam já estar processados
    e distribuídos da maneira que o Infogr.am espera. Por exemplo, um gráfico de
    linhas espera por uma tabela com: uma coluna com identificadores para cada
    linha do gráfico, e várias colunas com os valores associados a cada
    identificador ao longo do tempo. Isso fica mais claro na Figura
    \ref{fig:infogram-sample-spreadsheet}. O gráfico resultante está na Figura
    \ref{fig:infogram-sample-chart}.
    
    Para passar uma planilha para o Infogr.am, basta copiar a parte desejada
    dela em um editor de planilhas de sua escolha e colá-la no menu de edição do
    infográfico. Ou você pode usar a opção de carregar de um arquivo em seu
    computador. Note que se a planilha carregada não estiver com a disposição
    que o Infogr.am espera, o resultado será inesperado.
    
    \begin{figure}[ht]
      \centering
      \includegraphics[width=.9\linewidth]{infogram-sample-chart.png}
      \caption{
        \footnotesize
        \it
        Gráfico de linha resultante da planilha da Figura
        \ref{fig:infogram-sample-spreadsheet}.
      }
      \label{fig:infogram-sample-chart}
    \end{figure}

\clearpage
\section{Ferramentas de uso específico}
  Às vezes queremos um tipo mais específico de Infográfico, e serviços Web de
  propósito geral nesse aspecto podem não ter tudo que a gente quer nesses
  casos. Essa seção dedica-se a explicar alguns outro sites de Infográficos de
  uso mais específico, como geradores especializados de ``Word Clouds'', por
  exemplo.

  \subsection{Google Trends}
    \textbf{Endereço: \url{http://www.google.com/trends/}}
    
    Google Trends é um site do Google que fornece estatísticas e infográficos
    interativos sobre chaves de busca usadas pelas pessoas no site principal
    deles. Por exemplo, você pode pedir as Trends da busca ``\textit{cats}'', e
    o Google Trends mostrará uma página com diversas estatísticas sobre essa
    busca, como frequência ao longo do tempo, distribuição geográfica e termos
    relacionados.
    
    \begin{figure}[ht]
      \centering
      \includegraphics[width=.9\linewidth]{cats.png}
      \caption{
        \footnotesize
        \it
        Infográfico de ``Interesse ao longo do tempo'' da chave de busca
        ``cats'', pelo Google Trends.
      }
      \label{fig:cats}
    \end{figure}
    
    Cada um dos infográficos na página apresentada tem uma opção ``Embed''. Ela
    abre uma janela para você configurar o tamanho que você deseja que o
    infográfico tenha na sua página (em pixels) e fornece o código HTML
    correspondente. O uso do Google Trends é bem direto e específico mesmo.

  \subsection{Wordle}
    \textbf{Endereço: \url{http://www.wordle.net/}}
    
    Wordle é um serviço Web de geração de ``Word Clouds'' (nuvens de palavras).
    É um tipo específico de infográfico onde o que se quer mostrar é quais
    palavras (em um dado conjunto delas) são mais usadas, ou mais procuradas,
    ou mais importantes em algum aspecto.
    
    As palavras ficam visualmente distribuídas em uma mesma imagem e as mais
    ``relevantes'' ficam maiores que as outras. A distribuição delas pode seguir
    diversos padrões, fontes, cores e formatos. Algumas vezes elas assumem até o
    formato de algum objeto, podendo ter um objetivo mais artístico do que
    informacional.
    
    \begin{figure}[ht]
      \centering
      \includegraphics[width=.9\linewidth]{wordle-1.png}
      \caption{
        \footnotesize
        \it
        Exemplo de ``Word Cloud'' gerada pelo Wordle, usando termos do código
        LaTeX desta apostila.
      }
      \label{fig:wordle-1}
    \end{figure}
    
    Para usar o Wordle, não é necessário cadastro nenhum. Basta entrar no site e
    clicar em ``Create''. Ele pedirá para você fornecer um texto ou um
    \textit{link} para uma página Web que satisfaça certas condições (na dúvida,
    coloque e veja se funciona). Ele usará a frequência com que cada palavra
    aparece no texto ou no site para formular os \textit{pesos} delas na
    \textit{Word Cloud} resultante.
    
    Uma vez fornecidas as palavras, ele gerará um infográfico inicial usando
    configurações aleatórias, e você poderá mudar essas configurações para
    alterar a forma com que a \textit{Word Cloud} está apresentada. Para isso,
    basta acessar os menus na barra superior do visualizador do infográfico
    (pode estar razoavelmente difícil de enxergar devido às cores usadas). As
    opções nesses menus são bastante intuitivas, sendo as principais as dos
    menus ``Font'', ``Layout'' e ``Color''. Também é possível usar o botão
    ``Randomize'' para re-gerar o gráfico usando alguma outra configuração aleatória.
    
    \begin{figure}[ht]
      \centering
      \includegraphics[width=.9\linewidth]{wordle-2.png}
      \caption{
        \footnotesize
        \it
        Outro exemplo de ``Word Cloud'' gerada pelo Wordle, usando o mesmo
        conjunto de palavras.
      }
      \label{fig:wordle-2}
    \end{figure}
    
    Uma vez que o resultado lhe satisfizer, é só usar o botão ``Save to public
    gallery'' (salvar na galeria pública). Toda \textit{Word Cloud} feita no
    Wordle é pública, isso é, acessível a todos. No entanto, só o criador dela
    pode removê-la. Assim que você publicar sua imagem, aparecerá uma página
    explicando como se faz isso. Basicamente, você pode ou remover o gráfico
    imediatamente da galeria pública (no \textit{link} ``DELETE THIS WORDLE''),
    ou guardar o endereço URL fornecido pelo Wordle e remover sua \textit{Word
    Cloud} posteriormente.
    
    Por fim, ao final da página com a notificação de publicação, haverá o código
    HTML para fazer o \textit{embedding} do infográfico produzido no seu site.
    Note que se você remover seu trabalho da galeria pública, qualquer
    \textit{embedding} que você tenha feito dele não irá mais funcionar. Além
    disso, dependendo do tamanho que você queira que a \emph{Word Cloud}
    apresente no seu site, talvez valha mais a pena baixar uma imagem de alta
    resolução dela do que usar \emph{embedding}.

\end{document}
