\documentclass[brazil]{beamer}
\usepackage{beamerthemesplit}
\usepackage[brazilian]{babel}
\usepackage[utf8]{inputenc}
\usepackage{color}
\usepackage{xcolor}
\usepackage{amssymb}
\usepackage{amsmath}
\usepackage{fancybox}
\usepackage{ulem}
\usepackage{listings}
\usepackage{upquote}
\usetheme{JuanLesPins}
%\usetheme{Warsaw}
%\usetheme{CambridgeUS}
%\usetheme{Malmoe}


%\newcommand{\lyxline}[1][1pt]{%
%  \par\noindent%
%  \rule[.5ex]{\linewidth}{#1}\par}


\title{A Linguagem de Programação Lua}
\author{Wilson Kazuo Mizutani \\ kazuo@uspgamedev.org}

\begin{document}

\lstdefinelanguage{lua}
  {morekeywords={and,break,do,else,elseif,end,false,for,function,if,in,local,
     nil,not,or,repeat,return,then,true,until,while},
   morekeywords={[2]arg,assert,collectgarbage,dofile,error,_G,getfenv,
     getmetatable,ipairs,load,loadfile,loadstring,next,pairs,pcall,print,
     rawequal,rawget,rawset,select,setfenv,setmetatable,tonumber,tostring,
     type,unpack,_VERSION,xpcall},
   morekeywords={[2]coroutine.create,coroutine.resume,coroutine.running,
     coroutine.status,coroutine.wrap,coroutine.yield},
   morekeywords={[2]module,require,package.cpath,package.load,package.loaded,
     package.loaders,package.loadlib,package.path,package.preload,
     package.seeall},
   morekeywords={[2]string.byte,string.char,string.dump,string.find,
     string.format,string.gmatch,string.gsub,string.len,string.lower,
     string.match,string.rep,string.reverse,string.sub,string.upper},
   morekeywords={[2]table.concat,table.insert,table.maxn,table.remove,
   table.sort},
   morekeywords={[2]math.abs,math.acos,math.asin,math.atan,math.atan2,
     math.ceil,math.cos,math.cosh,math.deg,math.exp,math.floor,math.fmod,
     math.frexp,math.huge,math.ldexp,math.log,math.log10,math.max,math.min,
     math.modf,math.pi,math.pow,math.rad,math.random,math.randomseed,math.sin,
     math.sinh,math.sqrt,math.tan,math.tanh},
   morekeywords={[2]io.close,io.flush,io.input,io.lines,io.open,io.output,
     io.popen,io.read,io.tmpfile,io.type,io.write,file:close,file:flush,
     file:lines,file:read,file:seek,file:setvbuf,file:write},
   morekeywords={[2]os.clock,os.date,os.difftime,os.execute,os.exit,os.getenv,
     os.remove,os.rename,os.setlocale,os.time,os.tmpname},
   alsodigit = {.},
   sensitive=true,
   morecomment=[l]{--},
   morecomment=[s]{--[[}{]]},
   morestring=[b]",
   morestring=[d]',
   morestring=[s]{[[}{]]},
  }

%\lstset{language=c,
%        numbers=left,
%        numberstyle=\tiny,
%        showstringspaces=false,
%        aboveskip=-40pt,
%        frame=leftline
%        }
\lstset{
  language=lua,
  basewidth=0.5em,
  numbers=left,
  numbersep=5pt,
  basicstyle=\ttfamily\scriptsize,
  keywordstyle=\color{blue}\bfseries,
  keywordstyle=[2]\color{blue},
  commentstyle=\color[rgb]{0,.6,0},
  stringstyle=\color{red},
  showstringspaces=false,
  upquote=true,
  escapeinside={(*}{*)},
  escapebegin=\itshape,
}

\frame{\titlepage}

\frame{\tableofcontents}

%-------------------------------------
\section{Introdução e breve histórico}
%-------------------------------------
\frame{
  \begin{center}
  \LARGE 1. Introdução e breve histórico
  \end{center}
}
%-------------------------------------
\frame{
  \frametitle{Histórico}
  \begin{itemize}
    \item Projeto da PUC-Rio (Tecgraf e LabLua)
    \pause
    \item Site oficial: www.lua.org
    \pause
    \item Últimas versões:
    \pause
    \begin{itemize}
      \item 5.0 (obsoleto)
      \pause
      \item 5.1 (atual)
      \pause
      \item 5.2 (muito hipster ainda)
    \end{itemize}
  \end{itemize}
}
%-------------------------------------
\frame{
  \frametitle{Principais características}
  \begin{itemize}
    \pause
    \item Simplicidade
    \pause
    \item Eficiência
    \pause
    \item Versatilidade
    \pause
    \item Leveza
  \end{itemize}
}
%-------------------------------------
\begin{frame}[fragile]
  \frametitle{Exercício 1.1: Hello World!!!}
  \pause
  \begin{columns}
    \column[t]{.5\textwidth}
      \begin{block}{helloworld.lua}
        \begin{lstlisting}
-- My hello world Lua program:
print "Hello World!"
        \end{lstlisting}
      \end{block}
    \pause
    \column[t]{.5\textwidth}
      \begin{block}{Saída}
        \begin{verbatim}
Hello World!  \end{verbatim}
      \end{block}
  \end{columns}
\end{frame}
%-------------------------------------
\section{Expressões básicas}
%-------------------------------------
\frame{
  \begin{center}
  \LARGE 2. Expressões básicas
  \end{center}
}
%-------------------------------------
\subsection{2.1. Tipos escalares.}
%-------------------------------------
\frame{
  \begin{center}
  \Large 2.1. Tipos escalares
  \end{center}
}
%-------------------------------------
\frame{
  \frametitle{Tipagem Dinâmica}
  \begin{itemize}
    \pause
    \item Comum em linguagens de script.
    \pause
    \item O tipo de uma variável é definido pelo valor atual dela.
    \pause
    \item Portanto, o tipo de uma variável pode mudar ao longo do
          programa.
  \end{itemize}
}
%-------------------------------------
\begin{frame}[fragile]
  \frametitle{O tipo e valor nil}
  \pause
  \begin{columns}
    \column[t]{.5\textwidth}
      \begin{block}{nil.lua}
        \begin{lstlisting}
-- x has a nil value.
x = nil
print(x)
print(type(x))
        \end{lstlisting}
      \end{block}
    \pause
    \column[t]{.5\textwidth}
      \begin{block}{Saída:}
        \begin{verbatim}
nil
nil     \end{verbatim}
      \end{block}
  \end{columns}
\end{frame}
%-------------------------------------
\begin{frame}[fragile]
  \frametitle{O tipo e valor nil}
  \begin{center}E agora?\end{center}
  \begin{columns}
    \column[t]{.5\textwidth}
      \begin{block}{nil.lua}
        \begin{lstlisting}
-- x has a nil value.
-- x = nil
print(x)
print(type(x))
        \end{lstlisting}
      \end{block}
    \pause
    \column[t]{.5\textwidth}
      \begin{block}{Saída:}
        \begin{verbatim}
nil
nil     \end{verbatim}
      \end{block}
  \end{columns}
  \pause
  \begin{center}
    Importante! Variáveis não inicializadas (que "não existem")
    assumem automaticamente o valor de nil.
  \end{center}
\end{frame}
%-------------------------------------
\begin{frame}[fragile]
  \frametitle{Valores booleanos}
  \pause
  \begin{columns}
    \column[t]{.5\textwidth}
      \begin{block}{boolean.lua}
        \begin{lstlisting}
-- x and y have boolean values.
x = false
y = true
print(type(x), x)
print(type(y), y)
        \end{lstlisting}
      \end{block}
    \pause
    \column[t]{.5\textwidth}
      \begin{block}{Saída:}
        \begin{verbatim}
boolean   false
boolean   true  \end{verbatim}
      \end{block}
  \end{columns}
\end{frame}
%-------------------------------------
\begin{frame}[fragile]
  \frametitle{Valores numéricos}
  \pause
  \begin{columns}
    \column[t]{.5\textwidth}
      \begin{block}{number.lua}
        \begin{lstlisting}
-- x and y have number values.
x = 42
y = .9e4 + 1
print(type(x), x)
print(type(y), y)
        \end{lstlisting}
      \end{block}
    \pause
    \column[t]{.5\textwidth}
      \begin{block}{Saída:}
        \begin{verbatim}
number  42
number  9001  \end{verbatim}
      \end{block}
  \end{columns}
\end{frame}
%-------------------------------------
\begin{frame}[fragile]
  \frametitle{Cadeias de caracteres (string)}
  \pause
  \begin{columns}
    \column[t]{.5\textwidth}
      \begin{block}{string.lua}
        \begin{lstlisting}
-- x and y have string values.
x = "hello"
y = [[
This is a multi-lined
text, which is very common
in scripting languages.]]
print(type(x), x)
print(type(y))
print(y)
        \end{lstlisting}
      \end{block}
    \pause
    \column[t]{.5\textwidth}
      \begin{block}{Saída:}
        \begin{verbatim}
string  hello
string
This is a multi-lined
text, which is very common
in scripting languages.  \end{verbatim}
      \end{block}
  \end{columns}
\end{frame}
%-------------------------------------
\subsection{2.2. Operações básicas}
%-------------------------------------
\frame{
  \begin{center}
  \Large 2.2. Operações básicas
  \end{center}
}
%-------------------------------------
\begin{frame}[fragile]
  \frametitle{Atribuição}
  \pause
  \begin{columns}
    \column[t]{.5\textwidth}
      \begin{block}{assignment.lua}
        \begin{lstlisting}
-- Simple assignment.
x = 10
y = 20
-- Complex assignment
a, b, c = true, "sometext", 99
print(a, b, c)
-- Cross assignment. What happens?
x, y = y, x
print(x, y)
        \end{lstlisting}
      \end{block}
    \pause
    \column[t]{.5\textwidth}
      \begin{block}{Saída 1:}
        \begin{verbatim}
true  sometext  99 \end{verbatim}
      \end{block}
      \pause
      \begin{block}{Saída 2:}
        \begin{verbatim}
20 10 \end{verbatim}
      \end{block}
  \end{columns}
  \pause
  \begin{center}
    Variáveis em excesso ficam com 'nil'. \\
    Valores em excesso são ignorados.
  \end{center}
\end{frame}
%-------------------------------------
\begin{frame}[fragile]
  \frametitle{Operadores aritméticos e concatenação}
  \pause
  \begin{columns}
    \column[t]{.5\textwidth}
      \begin{block}{arithmetic.lua}
        \begin{lstlisting}
-- Simple math
x = (10 * (42 + 7)) / (5 * 7 + 192)
-- Also power and module
y = 6.022 * 10 ^ 23
z = 19238479127491 % 5
        \end{lstlisting}
      \end{block}
    \pause
    \column[t]{.5\textwidth}
      \begin{block}{concatenation.lua}
        \begin{lstlisting}
-- String concatenation.
s = "Hakuna".." Matata"
length = #"heeey"
        \end{lstlisting}
      \end{block}
  \end{columns}
  \pause
  \begin{center}
    É possível usar variáveis no lugar dos valores.
  \end{center}
\end{frame}
%-------------------------------------
\begin{frame}[fragile]
  \frametitle{Conversão implícita entre numbers e strings}
  \pause
  \begin{columns}
    \column[t]{.5\textwidth}
      \begin{block}{conversion.lua}
        \begin{lstlisting}
-- x has a number value and
-- y has a string value.
x = 13
y = "37"
-- what about z and w?
z = x + y
w = x .. y
print(type(z), z)
print(type(w), w)
        \end{lstlisting}
      \end{block}
    \pause
    \column[t]{.5\textwidth}
      \begin{block}{Saída:}
        \begin{verbatim}
number  50
string  1337  \end{verbatim}
      \end{block}
  \end{columns}
\end{frame}
%-------------------------------------
%% TODO: exercício 2.1
%-------------------------------------
\begin{frame}[fragile]
  \frametitle{Operadores relacionais e lógicos}
  \pause
  \begin{columns}
    \column[t]{.5\textwidth}
      \begin{block}{relational.lua}
        \begin{lstlisting}
-- Relational operators always
-- return boolean values
less, notless = (x < y), (y >= x);
equal = (x == y)
different = (x ~= y)
        \end{lstlisting}
      \end{block}
    \pause
    \column[t]{.5\textwidth}
      \begin{block}{logical.lua}
        \begin{lstlisting}
-- 'not' inverts boolean evaluation
nottrue = not true
-- If x evaluates true, returns y,
-- else returns x.
condition1 = x and y
-- If x evaluates true, return x,
-- else returns y.
condition2 = x or y
        \end{lstlisting}
      \end{block}
  \end{columns}
\end{frame}
%-------------------------------------
\section{Expressões de controle}
%-------------------------------------
\frame{
  \begin{center}
  \LARGE 3. Expressões de controle
  \end{center}
}
%-------------------------------------
\subsection{3.1. Estruturas}
%-------------------------------------
\frame{
  \begin{center}
  \Large 3.1. Estruturas
  \end{center}
}
%-------------------------------------
\begin{frame}[fragile]
  \frametitle{Estrutura if..then..else }
  \pause
  \begin{columns}
    \column[t]{.5\textwidth}
      \begin{block}{ifthenelse.lua}
        \begin{lstlisting}
-- Condition statement.
if x < 10 then
  -- do something
elseif x < 20 then
  -- do something else
else
  -- do yet another something
end 
        \end{lstlisting}
      \end{block}
    \pause
    \column[t]{.5\textwidth}
      \begin{block}{vartest.lua}
        \begin{lstlisting}
-- A variable evalues to true if
-- and only if its value is neither
-- nil nor false.
if x then
  -- x is neither nil nor false
else
  -- x is nil or false
end
        \end{lstlisting}
      \end{block}
  \end{columns}
  \pause
  \begin{center}
    Zero não é avaliado para falso!
  \end{center}
\end{frame}
%-------------------------------------
%% TODO: exercício
%-------------------------------------
\begin{frame}[fragile]
  \frametitle{Estruturas while e repeat..until}
  \pause
  \begin{columns}
    \column[t]{.5\textwidth}
      \begin{block}{while.lua}
        \begin{lstlisting}
-- Loops while the condition is
-- true.
i = 1
while i <= 10 do
  -- something that happens
  -- 10 times
  i = i + 1
end
-- Loops forever... not.
while true do
  -- how many times?
  i = i - 1
  break
end
        \end{lstlisting}
      \end{block}
    \pause
    \column[t]{.5\textwidth}
      \begin{block}{repeatuntil.lua}
        \begin{lstlisting}
-- Loops until the condition is
-- true.
i = 1
repeat
  -- something that happens
  -- 10 times
  i = i + 1
until i == 10
-- Loops foerever... neither.
repeat
  -- and now, how many times?
  i = i - 1
  if i < 5 then break end
until false
        \end{lstlisting}
      \end{block}
  \end{columns}
\end{frame}
%-------------------------------------
\begin{frame}[fragile]
  \frametitle{Estrutura for}
  \pause
  \begin{columns}
    \column[t]{.5\textwidth}
      \begin{block}{for.lua}
        \begin{lstlisting}
-- Loops from i = 1 to i = 5
for i = 1,5 do
  print(i)
end

-- Loops from i = 1 to i = 5
-- jumping by 2 at a time.
for i = 1,5,2 do
  print(i)
end
        \end{lstlisting}
      \end{block}
    \pause
    \column[t]{.5\textwidth}
      \begin{block}{Saída 1:}
        \begin{verbatim}
1
2
3
4
5 \end{verbatim}
      \end{block}
      \pause
      \begin{block}{Saída 2:}
        \begin{verbatim}
1
3
5 \end{verbatim}
      \end{block}
  \end{columns}
\end{frame}
%-------------------------------------
%% TODO: exercício
%% sugestão: fazer um mesmo loop usando while, repeat e for.
%-------------------------------------
\subsection{3.2. Escopo.}
%-------------------------------------
\frame{
  \begin{center}
  \Large 3.2. Escopo
  \end{center}
}
%-------------------------------------
\begin{frame}[fragile]
  \frametitle{Escopo global}
  \pause
  \begin{center}
    Por padrão, as variáveis são todas globais.
  \end{center}
  \pause
  \begin{columns}
    \column[t]{.5\textwidth}
      \begin{block}{scope1.lua}
        \begin{lstlisting}
-- Changing the value of x
x = 0
if true then
  x = x + 1
end
print(x)
        \end{lstlisting}
      \end{block}
      \pause
      \begin{block}{Saída 1:}
        \begin{verbatim}
1 \end{verbatim}
      \end{block}
    \pause
    \column[t]{.5\textwidth}
      \begin{block}{scope2.lua}
        \begin{lstlisting}
-- Does x exists?
if true then
  x = 1
end
print(x)
        \end{lstlisting}
      \end{block}
      \pause
      \begin{block}{Saída 2:}
        \begin{verbatim}
1 \end{verbatim}
      \end{block}
  \end{columns}
\end{frame}
%-------------------------------------
\begin{frame}[fragile]
  \frametitle{Escopo local}
  \pause
  \begin{center}
    É preciso explicitar as variáveis locais.
  \end{center}
  \pause
  \begin{columns}
    \column[t]{.5\textwidth}
      \begin{block}{local.lua}
        \begin{lstlisting}
-- x will not be a global variable.
local x = 0
y = 5
do
  local x = 10
  do
    y = 20
    print(x, y)
  end
end
print(x, y)
        \end{lstlisting}
      \end{block}
    \pause
    \column[t]{.5\textwidth}
      \begin{block}{Saída 1:}
        \begin{verbatim}
10  20\end{verbatim}
      \end{block}
      \pause
      \begin{block}{Saída 2:}
        \begin{verbatim}
0   20\end{verbatim}
      \end{block}
  \end{columns}
\end{frame}
%-------------------------------------
%% TODO: exercício "o i do for é global ou local?"
%-------------------------------------
\section{Funções}
%-------------------------------------
\frame{
  \begin{center}
  \LARGE 4. Funções
  \end{center}
}
%-------------------------------------
\subsection{4.1. Criando e usando funções}
%-------------------------------------
\frame{
  \begin{center}
  \Large 4.1. Criando e usando funções
  \end{center}
}
%-------------------------------------
\begin{frame}[fragile]
  \frametitle{O tipo function}
  \pause
  \begin{center}
    Função é um tipo nativo em Lua!
  \end{center}
  \pause
  \begin{columns}
    \column[t]{.5\textwidth}
      \begin{block}{function.lua}
        \begin{lstlisting}
-- f has a function value
f = function() end
print(type(f))
print(f)
        \end{lstlisting}
      \end{block}
    \pause
    \column[t]{.5\textwidth}
      \begin{block}{Saída:}
        \begin{verbatim}
function
function: 0x9d74b0  \end{verbatim}
      \end{block}
      \pause
      Não é um tipo escalar.
  \end{columns}
\end{frame}
%-------------------------------------
\begin{frame}[fragile]
  \frametitle{Definindo e usando funções}
  \pause
  \begin{columns}
    \column[t]{.5\textwidth}
      \begin{block}{functiondef.lua}
        \begin{lstlisting}
-- f takes x and returns x+10
f = function(x)
  return x + 10
end
-- Another way to do it:
function g(x)
  if x < 10 then
    print "Less than 10"
  else
    print "Greater than 10"
  end
  -- this one returns nothing
end
        \end{lstlisting}
      \end{block}
    \pause
    \column[t]{.5\textwidth}
      \begin{block}{functionusage.lua}
        \begin{lstlisting}
-- The result is 15
result = f(5)
-- Just calling
g(15)
-- When passing a single string
-- argument, parenthesis can be
-- omitted.
print "single string arg"
        \end{lstlisting}
      \end{block}
      \pause
      Argumentos são locais.
  \end{columns}
\end{frame}
%-------------------------------------
%% TODO exercício
%-------------------------------------
\begin{frame}[fragile]
  \frametitle{Definindo e usando funções}
  \pause
  \begin{center}
    Quantidade de parâmetros e retornos
  \end{center}
  \pause
  \begin{columns}
    \column[t]{.5\textwidth}
      \begin{block}{functionvardef.lua}
        \begin{lstlisting}
-- Functions may receive a
-- variable number of arguments
function f(x,y,z,...)
  local a, b, c = ...
  return a*x + b*y + c*z
end
-- And they may return many
-- values aswell.
function g()
  return "hey","listen"
end
-- Or no value at all.
function h() end
        \end{lstlisting}
      \end{block}
    \pause
    \column[t]{.5\textwidth}
      \begin{block}{functionvarusage.lua}
        \begin{lstlisting}
-- Using variable args.
f(1,2,3) -- error
f(1,2,3,4,5,6) -- ok
f() -- error, but possible
-- Getting multipe returns.
a,b = g() -- ok
c = g() -- "listen" is lost
d = h() -- guess what? nil.
        \end{lstlisting}
      \end{block}
  \end{columns}
\end{frame}
%-------------------------------------
%% TODO exercício
%-------------------------------------
\subsection{4.2. Upvalues}
%-------------------------------------
\frame{
  \begin{center}
  \Large 4.2. Upvalues
  \end{center}
}
%-------------------------------------
\begin{frame}[fragile]
  \frametitle{I heard you like functions...}
  \pause
  \begin{columns}
    \column[t]{.5\textwidth}
      \begin{block}{upvalue.lua}
        \begin{lstlisting}
-- Mindblow function
function yodog (x)
  local upvalue = x
  local function result ()
    return upvalue
  end
  return result
end
-- Make a few copies
f = yodog(10)
g = yodog("fly me to the moon")
print(f())
print(g())
        \end{lstlisting}
      \end{block}
    \pause
    \column[t]{.5\textwidth}
      \begin{block}{Saída:}
        \begin{verbatim}
10
fly me to the moon \end{verbatim}
      \end{block}
  \end{columns}
\end{frame}
%-------------------------------------
%% TODO exercício
%% - a variável "upvalue" era necessária?
%-------------------------------------
\section{Tabelas}
%-------------------------------------
\frame{
  \begin{center}
  \LARGE 5. Tabelas
  \end{center}
}
%-------------------------------------
\subsection{5.1. Criando e manipulando tabelas}
%-------------------------------------
\frame{
  \begin{center}
  \Large 5.1. Criando e manipulando tabelas
  \end{center}
}
%-------------------------------------
\begin{frame}[fragile]
  \frametitle{O tipo table}
  \pause
  \begin{center}
    Tabelas, assim como funções, formam um tipo nativo do Lua
    que não é escalar
  \end{center}
  \pause
  \begin{columns}
    \column[t]{.5\textwidth}
      \begin{block}{table.lua}
        \begin{lstlisting}
-- t has a table value
t = {}
print(type(t))
print(t)
        \end{lstlisting}
      \end{block}
    \pause
    \column[t]{.5\textwidth}
      \begin{block}{Saída:}
        \begin{verbatim}
table
table: 0x866780  \end{verbatim}
      \end{block}
  \end{columns}
\end{frame}
%-------------------------------------
\begin{frame}[fragile]
  \frametitle{Criando uma tabela com campos}
  \pause
  \begin{columns}
    \column[t]{.5\textwidth}
      \begin{block}{tablecreation.lua}
        \begin{lstlisting}
-- t receives a table initialized
-- with some fields.
t = {
  10, 20, 30,
  a = 40, b = 50,
  [3.14159] = "PI"
}
-- Print t's contents.
table.foreach(t, print)
        \end{lstlisting}
      \end{block}
    \pause
    \column[t]{.5\textwidth}
      \begin{block}{Saída:}
        \begin{verbatim}
1       10
2       20
3       30
a       40
3.14159 PI
b       50  \end{verbatim}
      \end{block}
  \end{columns}
\end{frame}
%-------------------------------------
\begin{frame}[fragile]
  \frametitle{Inserindo campos após a criação}
  \pause
  \begin{columns}
    \column[t]{.5\textwidth}
      \begin{block}{tableinsertion.lua}
        \begin{lstlisting}
-- t receives an empty table
t = {}
-- Inserting some fields.
t[1] = true
t[0.999] = "almost one"
t["a function!"] = function() end
-- Syntax sugar for string keys:
t.somefield = {}
-- Print t's contents.
table.foreach(t, print)
        \end{lstlisting}
      \end{block}
    \pause
    \column[t]{.5\textwidth}
      \begin{block}{Saída:}
        \begin{verbatim}
1           true
a function! function: ...
0.999       almost one
somefield   table: ...    \end{verbatim}
      \end{block}
  \end{columns}
\end{frame}
%-------------------------------------
\begin{frame}[fragile]
  \frametitle{Removendo campos}
  \pause
  \begin{columns}
    \column[t]{.5\textwidth}
      \begin{block}{tableremoval.lua}
        \begin{lstlisting}
-- t receives a non-empty table.
t = { 10, 20, 30 }
-- Let's check it out.
table.foreach(t, print)
-- Removing middle field.
t[2] = nil
-- What about now?
table.foreach(t, print)
        \end{lstlisting}
      \end{block}
    \pause
    \column[t]{.5\textwidth}
      \begin{block}{Saída 1:}
        \begin{verbatim}
1 10
2 20
3 30    \end{verbatim}
      \end{block}
      \begin{block}{Saída 2:}
        \begin{verbatim}
1 10
3 30    \end{verbatim}
      \end{block}
  \end{columns}
  \pause
  \begin{center}
    É o mesmo de antes: campos que valem 'nil' não existem!
  \end{center}
\end{frame}
%-------------------------------------
\begin{frame}[fragile]
  \frametitle{Acessando campos pela chave}
  \pause
  \begin{columns}
    \column[t]{.5\textwidth}
      \begin{block}{tableaccess.lua}
        \begin{lstlisting}
-- Let's go back to the table:
t = {
  10, 20, 30,
  a = 40, b = 50,
  [3.14159] = "PI"
}
-- Access fields
print(t[1], t[2], t[3])
print(t["a"], t["b"], t[3.14159])
-- Also works
print(t.a, t.b)
        \end{lstlisting}
      \end{block}
    \pause
    \column[t]{.5\textwidth}
      \begin{block}{Saída 1:}
        \begin{verbatim}
10  20  30 \end{verbatim}
      \end{block}
      \pause
      \begin{block}{Saída 2:}
        \begin{verbatim}
40  50  PI \end{verbatim}
      \end{block}
      \pause
      \begin{block}{Saída 3:}
        \begin{verbatim}
40  50 \end{verbatim}
      \end{block}
  \end{columns}
\end{frame}
%-------------------------------------
\begin{frame}[fragile]
  \frametitle{Acessando campos usando for}
  \pause
  \begin{columns}
    \column[t]{.5\textwidth}
      \begin{block}{tablefor.lua}
        \begin{lstlisting}
-- Same table again:
t = {
  10, 20, 30,
  a = 40, b = 50,
  [3.14159] = "PI"
}
-- All fields
for k,v in pairs(t) do
  print(k,v)
end
-- Indexed fields
for i,v in ipairs(t) do
  print(i,v)
end
        \end{lstlisting}
      \end{block}
    \pause
    \column[t]{.5\textwidth}
      \begin{block}{Saída 1:}
        \begin{verbatim}
1       10
2       20
3       30
a       40
3.14159 PI
b       50  \end{verbatim}
      \end{block}
      \pause
      \begin{block}{Saída 2:}
        \begin{verbatim}
1       10
2       20
3       30 \end{verbatim}
      \end{block}
  \end{columns}
\end{frame}
%-------------------------------------
\begin{frame}[fragile]
  \frametitle{Acessando campos usando foreach}
  \pause
  \begin{columns}
    \column[t]{.5\textwidth}
      \begin{block}{tableforeach.lua}
        \begin{lstlisting}
-- He likes us:
t = {
  10, 20, 30,
  a = 40, b = 50,
  [3.14159] = "PI"
}
-- All fields
table.foreach(t, print)
-- Indexed fields
table.foreachi(t, print)
        \end{lstlisting}
      \end{block}
    \pause
    \column[t]{.5\textwidth}
      \begin{block}{Saída 1:}
        \begin{verbatim}
1       10
2       20
3       30
a       40
3.14159 PI
b       50  \end{verbatim}
      \end{block}
      \pause
      \begin{block}{Saída 2:}
        \begin{verbatim}
1       10
2       20
3       30 \end{verbatim}
      \end{block}
  \end{columns}
\end{frame}
%-------------------------------------
\begin{frame}[fragile]
  \frametitle{Tamanho da tabela}
  \pause
  \begin{center}
    Funciona apenas para campos indexados
  \end{center}
  \pause
  \begin{columns}
    \column[t]{.5\textwidth}
      \begin{block}{tablelength.lua}
        \begin{lstlisting}
-- One last time?
t = {
  10, 20, 30,
  a = 40, b = 50,
  [3.14159] = "PI"
}
-- Table length
print(#t)
        \end{lstlisting}
      \end{block}
    \pause
    \column[t]{.5\textwidth}
      \begin{block}{Saída:}
        \begin{verbatim}
3 \end{verbatim}
      \end{block}
  \end{columns}
\end{frame}
%-------------------------------------
%% TODO exercício
%% - sugestão: construir "tabuada"
%% - desafio: qual a diferença entre
%%   usar pairs e usar foreach?
%-------------------------------------
\begin{frame}[fragile]
  \frametitle{Possibilidades e restrições}
  \begin{itemize}
    \pause
    \item Praticamente QUALQUER COISA pode ser inserido em uma
          tabela de Lua, seja como chave ou como valor.
    \pause
    \item Isso significa que tabelas podem ter outras tabelas e
          até mesmo funçoes como chaves e valores!
    \pause
    \item A ÚNICA EXCEÇÃO é o \verb$nil$:
    \begin{itemize}
      \pause
      \item Chaves NUNCA podem ser \verb$nil$ (isso causa um erro)
      \pause
      \item Campos com valores \verb$nil$ são removidos da tabela.
    \end{itemize}
  \end{itemize}
\end{frame}
%-------------------------------------
\begin{frame}[fragile]
  \frametitle{Tabelas são referências}
  \pause
  \begin{center}
    Veja o que acontece quando passamos uma tabela como
    argumento de uma função que tenta mudar ela
  \end{center}
  \pause
  \begin{columns}
    \column[t]{.5\textwidth}
      \begin{block}{tableisreference.lua}
        \begin{lstlisting}
-- Starts with x = 20
t = { x = 20 }
print(t.x)
-- This changes it
function change (ref)
  ref.x = "not"
end
change(t)
print(t.x)
        \end{lstlisting}
      \end{block}
    \pause
    \column[t]{.5\textwidth}
      \begin{block}{Saída:}
        \begin{verbatim}
20
not \end{verbatim}
      \end{block}
  \end{columns}
\end{frame}
%-------------------------------------
\subsection{5.2. Atalhos}
%-------------------------------------
\frame{
  \begin{center}
  \Large 5.2. Atalhos
  \end{center}
}
%-------------------------------------
%% TODO add to table part
%%-- The same goes for single table
%%-- arguments.
%%ls { "first", "second", "third" }


%-------------------------------------
%-------------------------------------

\begin{frame}[fragile]
  \frametitle{Definindo e usando funções}
  \pause
  \begin{center}
    Métodos!
  \end{center}
  \pause
  \begin{columns}
    \column[t]{.5\textwidth}
      \begin{block}{method1.lua}
        \begin{lstlisting}
-- A method is a function whithing
-- a table that takes the table
-- itself as its first argument:
t = { "a", "b", "c" }
function t.ls(self)
  table.foreach(self, print)
end
-- But then we have to do this:
t.ls(t)
-- Which is annoying. Lua has
-- a syntax sugar for that:
t:ls()
        \end{lstlisting}
      \end{block}
    \pause
    \column[t]{.5\textwidth}
      \begin{block}{method2.lua}
        \begin{lstlisting}
-- The syntax sugar also works
-- for method definitions, in
-- which case the first argument
-- will be implicit, being
-- called 'self'
function t:size()
  return table.getn(self)
end
        \end{lstlisting}
      \end{block}
  \end{columns}
\end{frame}

%-------------------------------------
\subsection{2.5. Regras de visibilidade}
%-------------------------------------

\frame{
  \frametitle{Blocos}
  \begin{itemize}
    \pause
    \item Um script lua apresenta uma hierarquia de blocos.
    \pause
    \item Um bloco é uma sequência de comandos, como os
          exemplos que vimos até agora.
    \pause
    \item Um bloco pode ter outros blocos dentro de si,
          através de:
    \begin{itemize}
      \pause
      \item estruturas de controle (if, while, for, etc);
      \pause
      \item definições de funções; ou
      \pause
      \item uso explícito de "do..end".
    \end{itemize}
  \end{itemize}
}

%-------------------------------------

\frame{
  \frametitle{Visibilidade}
  \begin{itemize}
    \pause
    \item Variáveis globais são visíveis a partir de
          qualquer bloco, contanto que não esteja
          obscurecida por alguma variável local visível
          de mesmo nome.
    \pause
    \item Variáveis locais são visíveis no bloco em que
          foram declaradas e nos blocos aninhados nele,
          contanto que sejam referenciadas após a linha em
          que foram definidas.
    \pause
    \item Se um bloco tenta acessar uma variável que não
          está visível de jeito nenhum para ele, ele
          receberá um 'nil' na cara.
  \end{itemize}
}

%-------------------------------------
\section{Bibliotecas básicas}
%-------------------------------------

\frame{
  \begin{center}
    \LARGE 3. Bibliotécas básicas
  \end{center}
}

%-------------------------------------
\section{Unlimited slide works}
%-------------------------------------
%-------------------------------------
\frame{
  \frametitle{Threads}
  \begin{itemize}
    \pause
    \item Threads se parecem com funções, só que elas podem ser
          executadas sincronizadamente (i.e. paralelamente ao
          mesmo tempo)
    \pause
    \item Talvez não pareça, mas isso é bem complicado de usar!
    \pause
    \item Por isso não abordarei threads nesse curso. Os curiosos
          podem fuçar o manual do Lua e explodirem seus
          computadores à vontade =P.
  \end{itemize}
}
%-------------------------------------

\frame{
  \frametitle{Userdata}
  \begin{itemize}
    \pause
    \item Userdata representam objetos nativos que vieram de
          fora do Lua.
    \pause
    \item Portanto, não podem ser criados de dentro dele.
    \pause
    \item E sua forma de uso é totalmente determinada por quem
          o criou.
    \pause
    \item Por exemplo, ele poderia ser um objeto associado a uma
          imagem, e ter métodos "draw(x,y)" ou "rotate(degrees)".
  \end{itemize}
}

\begin{frame}
  \frametitle{Tópicos avançados que não aparecem nos slides}
  \begin{itemize}
    \item Threads.
    \item Ambientes de funções.
    \item Metatabelas.
    \item API C.
  \end{itemize}
\end{frame}

%-------------------------------------

\begin{frame}
  \begin{center}
    \LARGE Obrigado!
  \end{center}
  \begin{itemize}
    \item Visitem www.uspgamedev.org
    \item Dúvidas: kazuo@uspgamedev.org
    \item Se seu interesse for desenvolvimento de games,
          vale MUITO a pena das uma conferida na engine
          LÖVE.
    \begin{itemize}
      \item Site oficial: www.love2d.org
      \item Jogo feito em LÖVE: http://stabyourself.net/mari0
    \end{itemize}
  \end{itemize}
\end{frame}

%-------------------------------------

\end{document}

