
\documentclass[12pt,onecolumn]{article}
\usepackage[brazilian]{babel}
\usepackage[utf8]{inputenc}
\usepackage{hyperref}
\usepackage[section]{placeins}
\usepackage{graphicx}
\usepackage{caption}
\usepackage{subcaption}
\usepackage{float}
\usepackage{framed,color}

\begin{document}

% Title page.
\begin{titlepage}

    % Title info.
    \title{
        \bf
        \LARGE Apostila de \\
        \Huge  Gimp
    }
    
    \author{Renan Teruo Carneiro \\ Wilson Kazuo Mizutani}
    
    % Print title.
    \maketitle
    
    % No numbering on this page.
    \thispagestyle{empty}
    
\end{titlepage}

% Print table of contents.
\tableofcontents

% Page break.
\clearpage

\section{Sobre o Gimp}
  O Gimp é legal.

\section{Instalando}
  Baixa e clique avançar-avançar-indução seja feliz.
  Ou sudo apt-get install.

\section{Introdução}
  Como uma introdução ao Gimp, vamos fazer uma edição bem simples. O objetivo
  será colorir a imagem de um ícone preto-e-branco, como mostrado na Figura
  \ref{fig:intro}. A imagem original está disponível em
  
  \begin{center}
    \url{http://game-icons.net/lorc/original/beast-eye.html}      
  \end{center}
  
   O ideal é pegar a versão com maior resolução possível, em PNG.

  \begin{figure}
  \centering
  \begin{subfigure}{.5\textwidth}
    \centering
    \includegraphics[width=.7\linewidth]{beast-eye.png}
    \label{fig:ex1_before}
  \end{subfigure}%
  \begin{subfigure}{.5\textwidth}
    \centering
    \includegraphics[width=.7\linewidth]{draft00.png}
    \label{fig:ex1_after}
  \end{subfigure}
  \caption{Imagem antes e depois da edição}
  \label{fig:intro}
  \end{figure}
  
  Com esse exercício esperamos apresentar as funcionalidades básicas e as
  técnicas essenciais para se manipular imagens com o Gimp. Mas apesar do foco
  nesse momento ser passar os conceitos básicos para o aluno, também
  introduziremos algumas das principais ferramentas usadas em projetos Gimp.
  
  \subsection{Abrindo a imagem e convertendo para RGBA}
    O primeiro passo do exercício é dizer para o Gimp que queremos trabalhar em
    RGBA (isso é, com cores e transparência) ao invés de Grayscale, que é o
    formato no qual a imagem do ícone está. E vamos aproveitar esse passo para
    mostrar o básico de copiar e colar imagens com Gimp (que não é muito
    diferente do convencional).
    
    Começamos abrindo a imagem original indo em {\bf File $\rightarrow$ Open}
    ou digitando {\bf CTRL+O}.
    
    \begin{framed}
      Algo que facilita muito o uso do Gimp é a familiarização com os atalhos
      de teclado. Saber uma meia-dúzia de atalhos é o suficiente para tornar
      dobrar a eficiência de trabalho. 
    \end{framed}
    
    Do jeito que o Gimp carrega a imagem (em Grayscale) não é possível aplicar
    nenhuma cor aos pixeis. Isso significa que se selecionarmos, por exemplo,
    a {\bf Pencil Tool} (atalho {\bf N}), e mudarmos a cor para vermelho, ainda
    assim pintaremos apenas tons de cinza.
    
    \begin{figure}[h]
      \centering
      \includegraphics[width=.6\textwidth]{screenshots/00-pencil_and_color.png}
      \caption{Seleciondo a {\bf Pencil Tool} e mudando a cor}
      \label{fig:pencil_and_color}
    \end{figure}
    
    \begin{framed}
      Para mudar de cor no Gimp, basta clicar duas vezes no retângulo colorido
      no meio da {\bf Toolbox }, como mostra a Figura \ref{fig:pencil_and_color}.
    \end{framed}
    
    \begin{framed}
      Para desfazer uma ação Gimp, pode-se ir em {\bf Edit $\rightarrow$ Undo}
      ou digitar {\bf CTRL+Z}. É um comando {\it extremamente útil}, e o
      histórico de ações que o Gimp guarda é bastante grande. Use ele para
      desfazer quaisquer alterações que você possa ter feito na imagem original
      antes de seguir os próximos passos.
    \end{framed}
    
    Vamos, então, criar uma nova imagem que use RGBA ao invés de Grayscale. Para
    isso, clicamos em {\bf File $\rightarrow$ New} ou digitamos {\bf CTRL+N}. Um
    menu {\bf Create a New Image} aparecerá. Nele temos acesso à várias
    configurações da nova imagem, como a largura e a altura dela. Essas
    manteremos como estão (512x512 se você pegou a maior resolução disponível).
    O que nos interessa aqui é mudar o formato de cores e o suporte a
    transparência. Basta mudar essas configurações conforme a Figura
    \ref{fig:grayscale_to_RGBA}.
    
    \begin{figure}[H]
      \centering
      \includegraphics[scale=.55]{screenshots/00-grayscale_to_RGBA.png}
      \caption{Criando uma imagem RGBA}
      \label{fig:grayscale_to_RGBA}
    \end{figure}
    
    Uma vez feito isso, basta voltarmos para a imagem original, e:
    
    \begin{itemize}
      \item
        Selecionarmos ela inteira com {\bf Botão Direito do Mouse $\rightarrow$
        Selection $\rightarrow$ All} ou {\bf CTRL+A}.
      \item
        Copiarmos ela com {\bf Botão Direito do Mouse $\rightarrow$ Edit
        $\rightarrow$ Copy} ou {\bf CTRL+C}.
      \item
        Voltarmos para a imagem com RGBA que acabamos de criar.
      \item
        Colarmos o ícone orignal nela com {\bf Botão Direito do Mouse
        $\rightarrow$ Edit $\rightarrow$ Paste} ou {\bf CTRL+V}.
    \end{itemize}
    
    
  \subsection{Sobre seleções}
    Agora que temos nossa imagem pronta para ser manipulada, vamos falar sobre
    seleções no Gimp. Seleções são um conceito primordial para qualquer tipo de
    edição, pois {\bf toda ferramenta que não seja de seleção, só é aplicada
    sobre a seleção atual}. Quando não há nenhuma seleção, então a imagem toda
    é considerada selecionada (ou seja, as ferramentas serão aplicadas em todas
    as partes dela).
    
    Para entender isso melhor, podemos escollher a {\bf Rectangle Selection
    Tool} (atalho {\bf R}) e selecionar uma região qualquer da imagem
    clicando e arrastando o mouse, e depois soltando-o. Agora, se escolhermos,
    por exemplo, a {\bf Bucket Feel Tool} (atalho {\bf SHIFT+B}) e tentarmos
    usá-la, obteremos o resultado mostrado na Figura
    \ref{fig:selective_painting}.
  
    %% 1) Explicação sobre seleções. Só funciona editar se estiver selecionado.
    \begin{figure}[H]
      \centering
      \includegraphics[scale=0.6]{screenshots/01-selective_painting.png}
      \caption{As ferramentas trabalham apenas nas regiões selecionadas}
      \label{fig:selective_painting}
    \end{figure}
    %% 2) Ferramentas de seleção: retângulo, tesoura e varinha.
    \begin{figure}[H]
      \centering
      \begin{subfigure}{.5\textwidth}
        \centering
        \includegraphics[width=.7\linewidth]{screenshots/02-free_select.png}
        \label{fig:free_select}
      \end{subfigure}%
      \begin{subfigure}{.5\textwidth}
        \centering
        \includegraphics[width=.7\linewidth]{screenshots/03-fuzzy_select.png}
        %\caption{Depois}
        \label{fig:fuzzy_select}
      \end{subfigure}
      \caption{Ferramenta de seleção livre e varinha mágica}
      \label{fig:select_tools}
    \end{figure}
    %% 3) Operações de conjunto: inversão (CTRL+I), conjunção (SHIFT), disjunção
    %%    (CTRL+SHIFT) e diferença (CTRL).
    %% 4) Selecionar o olho e pintar a pupila.
    \begin{figure}[H]
      \centering
      \includegraphics[width=.5\textwidth]{screenshots/04-pupil.png}
      \caption{Pintamos somente a pupila graças ao funcionamento das seleções}
      \label{fig:pupil}
    \end{figure}
    %% 5) Mencionar como mover seleções.
        
  \subsection{Sobre camadas}
    %% 1) Vamos fazer uma camada para o olho. Selecione apenas o olho.
    %% 2) CTRL+X, CTR+V, colar em nova camada.
    \begin{figure}[H]
      \centering
      \includegraphics[width=.5\textwidth]{screenshots/05-pasted_layer.png}
      \caption{Seleções coladas ficam em uma camada flutuante}
      \label{fig:pasted_layer}
    \end{figure}
    %% 3) Camadas 101
    %% 4) Renomear camada nova (F2 ou "Edit Layer Attributes")
    %% 5) Redimensionar camada para o tamanho da imagem
    %% 6) Preencher o vazio na camada de baixo (usar balde)
    %% 7) Separar as estrias em outra camada também.
    \begin{figure}[H]
      \centering
      \includegraphics[width=.5\textwidth]{screenshots/07-layers.png}
      \caption{As três camadas no final}
      \label{fig:layers}
    \end{figure}
    
  \subsection{Finalizando}
    %% 1) Pintar as estrias (explicar o modo "fill selection" do balde)
    \begin{figure}[H]
      \centering
      \includegraphics[width=.5\textwidth]{screenshots/06-partial.png}
      \caption{Resultado quase pronto da edição}
      \label{fig:partial}
    \end{figure}
    %% 2) Preencher o fundo com a textura "Leather" usando balde.
    %% 3) Ajustar cores, tons, etc.
    \begin{figure}[H]
      \centering
      \includegraphics[width=.7\linewidth]{draft00.png}
      \caption{Resultado final}
      \label{fig:final}
    \end{figure}
    %% 4) Propor pupilas melhores.
    
\section{Ferramentas}

  \subsection{Seleção}
    \subsubsection{Retângulo}
      \begin{figure}[H]
        O ÍCONE
        \includegraphics{gimp-icons/stock-tool-rect-select-22.png}
        \label{fig:rectselect}
      \end{figure}
      Essa ferramenta seleciona uma área retângular.

      \subsubsection{Elipse}
      \begin{figure}[H]
        O ÍCONE
        \includegraphics{gimp-icons/stock-tool-ellipse-select-22.png}
        \label{fig:ellipseselect}
      \end{figure}
      Essa ferramenta seleciona uma área elíptica.

      \subsubsection{Livre}
      \begin{figure}[H]
        O ÍCONE
        \includegraphics{gimp-icons/stock-tool-free-select-22.png}
        \label{fig:freeselect}
      \end{figure}
      Essa ferramenta seleciona uma área livre.
      
      \subsubsection{Magic}
      \begin{figure}[H]
        O ÍCONE
        \includegraphics{gimp-icons/stock-tool-fuzzy-select-22.png}
        \label{fig:magicselect}
      \end{figure}
      Essa ferramenta seleciona uma área mágica.

      \subsubsection{Tesoura}
      \begin{figure}[H]
        O ÍCONE
        \includegraphics{gimp-icons/stock-tool-iscissors-22.png}
        \label{fig:scissorsselect}
      \end{figure}
      Essa ferramenta seleciona uma área tesoura.
      
      \subsubsection{Cor}
      \begin{figure}[H]
        O ÍCONE
        \includegraphics{gimp-icons/stock-tool-by-color-select-22.png}
        \label{fig:colorselect}
      \end{figure}
      Essa ferramenta seleciona uma área colorida.


  \subsection{Edição}
    \subsubsection{Blur}
      \begin{figure}[H]
        O ÍCONE
        \includegraphics{gimp-icons/stock-tool-blur-22.png}
        \label{fig:blur}
      \end{figure}
      Borra suas coisas

  \subsubsection{Bucket}
      \begin{figure}[H]
        O ÍCONE
        \includegraphics{gimp-icons/stock-tool-bucket-fill-22.png}
        \label{fig:bucket}
      \end{figure}
      Preenche suas coisas
    
    \subsubsection{Clone}
      \begin{figure}[H]
        O ÍCONE
        \includegraphics{gimp-icons/stock-tool-clone-22.png}
        \label{fig:clone}
      \end{figure}
      Clona suas coisas

    \subsubsection{Color Picker}
      \begin{figure}[H]
        O ÍCONE
        \includegraphics{gimp-icons/stock-tool-color-picker-22.png}
        \label{fig:color-picker}
      \end{figure}
      Escolhe suas coisas
      
    \subsubsection{Dodge}
      \begin{figure}[H]
        O ÍCONE
        \includegraphics{gimp-icons/stock-tool-dodge-22.png}
        \label{fig:dodge}
      \end{figure}
      Desvia de suas coisas

    \subsubsection{Eraser}
      \begin{figure}[H]
        O ÍCONE
        \includegraphics{gimp-icons/stock-tool-eraser-22.png}
        \label{fig:eraser}
      \end{figure}
      Apaga suas coisas

    \subsubsection{Heal}
      \begin{figure}[H]
        O ÍCONE
        \includegraphics{gimp-icons/stock-tool-heal-22.png}
        \label{fig:heal}
      \end{figure}
      Cura suas coisas

    \subsubsection{Move}
      \begin{figure}[H]
        O ÍCONE
        \includegraphics{gimp-icons/stock-tool-move-22.png}
        \label{fig:move}
      \end{figure}
      Move suas coisas

    \subsubsection{Paintbrush}
      \begin{figure}[H]
        O ÍCONE
        \includegraphics{gimp-icons/stock-tool-paintbrush-22.png}
        \label{fig:brush}
      \end{figure}
      Pinta suas coisas
      
    \subsubsection{Pencil}
      \begin{figure}[H]
        O ÍCONE
        \includegraphics{gimp-icons/stock-tool-pencil-22.png}
        \label{fig:pencil}
      \end{figure}
      Rabisca suas coisas
      
    \subsubsection{Rotate}
      \begin{figure}[H]
        O ÍCONE
        \includegraphics{gimp-icons/stock-tool-rotate-22.png}
        \label{fig:rotate}
      \end{figure}
      Gira suas coisas
      
    \subsubsection{Scale}
      \begin{figure}[H]
        O ÍCONE
        \includegraphics{gimp-icons/stock-tool-scale-22.png}
        \label{fig:scale}
      \end{figure}
      Aumenta suas coisas

    \subsubsection{Smudge}
      \begin{figure}[H]
        O ÍCONE
        \includegraphics{gimp-icons/stock-tool-smudge-22.png}
        \label{fig:smudge}
      \end{figure}
      Também borra suas coisas
      
    \subsubsection{Text}
      \begin{figure}[H]
        O ÍCONE
        \includegraphics{gimp-icons/stock-tool-text-22.png}
        \label{fig:text}
      \end{figure}
      Escreve suas coisas




  \subsection{Controle de Cor}

\end{document}
