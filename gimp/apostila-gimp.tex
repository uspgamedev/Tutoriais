
\documentclass[12pt,onecolumn]{article}
\usepackage[brazilian]{babel}
\usepackage[utf8]{inputenc}


\begin{document}

\title{Apostile de Gimp}
\author{Renan Teruo Carneiro \\ Wilson Kazuo Mizutani}
\maketitle

\section{Introducao ao \LaTeX}
    LaTeX é um programa de formatação de texto e uma expansão do programa
    TeX, criado por Donald Knuth. Mas o que é um programa de
    formatação de texto?

    A maioria dos processadores de texto cuidam de 4 estágios para preparar
    um texto:
    1. O texto entra no computador.
    2. O texto é formatado em linhas, paragráfos e páginas.
    3. O texto é impresso na tela.
    4. O texto é impresso.

    O LaTeX e o TeX se preocupao somente com o segundo estagio
    do processamento. Para formatar um texto usando o LaTeX, escrevemos
    o texto em um editor de texto e compilamos. A partir daí, o \LaTeX
    cuidára da formatação do texto.

\subsection{Tipos de Documentos}
    Os tipos mais comuns de documentos que o \LaTeX aceita são
    \begin{enumerate}
    \item Book
    \item Article
    \item Letter
    \item Report
    \end{enumerate}

\subsection{Fontes}

    Podemos mudar o tamanho e o estilo das fontes usando os seguintes
    comandos:
    \begin{itemize}
    \item \verb#\textrm{...}# \textrm{roman}
    \item \verb#\textsf{...}# \textsf{sans serif}
    \item \verb#\texttt{...}# \texttt{typewriter}
    \item \verb#\textmd{...}# \textmd{medium}
    \item \verb#\textbf{...}# \textbf{boldface}
    \item \verb#\textup{...}# \textup{upright}
    \item \verb#\textit{...}# \textit{italic}
    \item \verb#\textsl{...}# \textsl{slanted}
    \item \verb#\textsc{...}# \textsc{small cap}
    \end{itemize}

\section{Primeiro arquivo}
    Agora, vamos criar um arquivo simples no \LaTeX. Copie as
    linhas abaixo no seu editor de texto favorito e rode o
    compilador de \LaTeX para gerar o pdf.

    \begin{verbatim}
    \documentclass{article}
    \begin{document}
    Hello, world!
    \end{document}
    \end{verbatim}

    Os comandos nesse arquivo são:
    \begin{itemize}
    \item \verb#\documentclass{..}# é o comando
        que define qual o tipo de documento estamos escrevendo.
    \item \verb#\begin{...} e \end{...}# são os
        comandos que definem quando algo começa e termina. No nosso caso,
        eles definem quando o documento começa e termina.
    \end{itemize}

    Agora, vamos acrescentar novos elementos nesse documento,
    e transformá-lo em um documento real.

\section{Estrutura de um Documento}
\subsection{Modificações para o Documento}
    Vamos estudar mais detalhadamente a estrutura de um documento no \LaTeX.
    Já vimos acima o \verb#\documentclass[...]{...}#. Ele aceita várias
    opções para modificar o documento. Essas opções, que colocadas
    entre os colchetes, são:
    \begin{itemize}
    \item \verb#\documentclass[10pt]{article}# Assim, o tamanho das letras do
        documento é setado para 10pt.
    \item \verb#documentclass[letterpaper]{article}# Esse comando faz com que
        o \LaTeX molde o seu texto para ser impresso em papel de carta.
    \item \verb#documentclass[twocolumn]{article}# Esse aqui faz com que o
        texto seja dividido em duas colunas ( como nos dicionários ).
    \item \verb#documentclass[oneside]{article}# Quando o \LaTeX montar o
        texto para impressão, ele vai configurá-lo para imprimir somente
        nas páginas ímpares.
    \end{itemize}

\subsection{Modificações para Páginas}
    Podemos modificar todas as páginas de um documento, ou alguma página
    especifica, usando os comandos \verb#\pagestyle{...}# e o
    \verb#\thispagestyle{...}#, respectivamente.

    Os argumentos que eles aceitam são:
    \begin{itemize}
    \item plain - O cabeçálho fica vazio e o rodapé contêm o número da
    página. Esse é o padrao para o article,
    \item empty - Tanto o cabeçálho como o rodapé ficam vazios,
    \item headings - O rodapé fica vazio e o cabeçálho contem o número da
            página e o nome do capítulo ou seção ou subseção,
    \item myheadings - Igual ao headings, mas o que é mostrando no lugar do
            nome do capítulo pode ser configurado.
    \end{itemize}

    Para modificar o modo com que as páginas são númeradas, usamos o
    comando \verb#\pagenumbering{...}#. Seus argumentos são:
    \begin{itemize}
    \item arabic - Numerais arábicos
    \item roman - Numerais romanos em minúsculo,
    \item Roman - Numerais romanos em maiúsculo,
    \item alph - Numerais por extenso em inglês em minúsculo,
    \item Alph - Numerais por extenso em inglês em maiúsculo.
    \end{itemize}

\subsection{Atualizar Exemplo}
    Esse são só alguns exemplos do que podemos mudar nos nossos documentos.
    Agora, vamos ver como fica o nosso documento de teste se colocarmos
    as opções \emph{twocolumn} e \emph{a4paper} para o documento, opções
    \emph{headings} e \emph{Roman} para as páginas. E vamos
    incrementá-lo com dois paragráfos de texto.

    Para criar novos paragráfos no \LaTeX só precisamos deixar uma linha
    totalmente em branco entre os dois paragráfos.

\subsection{Título}
    O \LaTeX também cuida da criação do título do documento. Para isso
    usamos os seguintes comandos
    \begin{verbatim}
    \title{título}
    \author{autor}
    \date{data}
    \maketitle
    \end{verbatim}

    Note que é preciso usar o comando \verb#\maketitle# ou o título não
    será gerado.

\subsection{Resumo}
    Nos documentos do tipo article e report podemos gerar um resumo usando
    os comandos \verb#\begin{abstract}# e \verb#\end{abstract}#.
    Tudo o que estiver escrito dentro dessas tags vai virar o resumo do
    documento, e será posicionado entre o título e o resto do documento.

\subsection{Divisão do Documento}
    Quando escrevemos textos grandes ( como esse ), é bom criar seções e
    subções para deixar tudo organizado e para facilitar a busca de infor
    mações. O \LaTeX tem uma hierarquia de comandos para criar divisões:
    \begin{enumerate}
	\item \verb#\chapter#
    \item \verb#\section#
    \item \verb#\subsection#
    \item \verb#\subsubsection#
    \item \verb#\paragraph#
    \item \verb#\subparagraph#
    \end{enumerate}
    
    O \verb#\chapter# só pode ser usado no tipo book.

\subsection{Nova Atualização}
    Agora vamos colocar mais informação no nosso documento. Vamos criar duas
    seções com duas subseções cada. Vamos criar um resumo também.

\end{document}

