
\documentclass[12pt,onecolumn]{article}
\usepackage[brazilian]{babel}
\usepackage[utf8]{inputenc}
\usepackage{graphicx}
\usepackage{caption}
\usepackage{subcaption}
\usepackage{float}
\usepackage{hyperref}

\begin{document}

\title{Apostila de Gimp}
\author{Renan Teruo Carneiro \\ Wilson Kazuo Mizutani}
\maketitle

\section{Sobre o Gimp}
  O Gimp é legal.

\section{Instalando}
  Baixa e clique avançar->avançar->indução seja feliz.
  Ou sudo apt-get install.

\section{Exercícios}

  \subsection{Colorindo uma imagem em preto-e-branco}
    Insira motivação aqui.

    \begin{figure}[H]
    \centering
    \begin{subfigure}{.5\textwidth}
      \centering
      \includegraphics[width=.7\linewidth]{beast-eye.png}
      \caption{Antes}
      \label{fig:sub1}
    \end{subfigure}%
    \begin{subfigure}{.5\textwidth}
      \centering
      \includegraphics[width=.7\linewidth]{draft00.png}
      \caption{Depois}
      \label{fig:sub2}
    \end{subfigure}
    %\caption{A figure with two subfigures}
    \label{fig:test}
    \end{figure}

    A imagem original está disponível em (pegue a maior versão possível em PNG)
    \begin{center}
      \url{http://game-icons.net/lorc/original/beast-eye.html}      
    \end{center}

    \subsubsection{Abrindo a imagem e convertendo para RGBA}
    

  \subsection{Destacando com cores}
    oi2.

  \subsection{Adulteração de rostos}
    Oi?

  \subsection{Montagem}
    É da padaria?

\section{Glossário}

\end{document}
